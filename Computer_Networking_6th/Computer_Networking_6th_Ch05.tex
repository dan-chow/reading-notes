\section{The Link Layer: Links, Access Networks, and LANs}
\begin{itemize}

\item
There are two fundamentally different types of link-layer channels. The first type are broadcast channels, which connect multiple hosts in wireless LANs, satellite networks, and hybrid fiber-coaxial cable (HFC) access networks. The second type of link-layer channel is the point-to-point communication link, such as that often found between two routers connected by a long-distance link, or between a user's office computer and the nearby Ethernet switch to which it is connected.

\item
Possible services that can be offered by a link-layer protocol include:
\begin{itemize}
\item\textit{Framing.}
\item\textit{Link access.}
\item\textit{Reliable delivery.}
\item\textit{Error detection and correction.}
\end{itemize}

\item
For the most part, the link layer is implemented in a \textbf{network adapter}, also sometimes known as a \textbf{network interface card (NIC)}. At the heart of the network adapter is the link-layer controller, usually a single, special-purpose chip that implements many of the link-layer services (framing, link access, error detection, and so on).

\item
Perhaps the simplest form of error detection is the use of a single \textbf{parity bit}.

\end{itemize}