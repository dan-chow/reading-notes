\section{Computer Networks and the Internet}
\begin{itemize}

\item The Internet is a computer network that interconnects hundreds of millions of computing devices throughout the world.

\item Indeed, the term \textit{computer network} is beginning to sound a bit dated, given the many nontraditional devices that are being hooked up to the Internet. In Internet jargon, all of these devices are called \textbf{hosts} or \textbf{end systems}.

\item End systems are connected together by a network of \textbf{communication links} and \textbf{packet switches}.

\item Packet switches come in many shapes and flavors, but the two most prominent types in today's Internet are \textbf{routers} and \textbf{link-layer switches}.

\item End systems access the Internet through \textbf{Internet Service Providers (ISPs)}.

\item End systems, packet switches, and other pieces of Internet run \textbf{protocols} that control the sending and receiving of information within the Internet. The \textbf{Transmission Control Protocol (TCP)} and the \textbf{Internet Protocol (IP)} are two of the most important protocols in the Internet.

\item \textbf{Internet standards} are developed by the Internet Engineering Task Force (IETF). The IETF standards documents are called \textbf{requests for comments (RFCs)}.

\item We can also describe the Internet from an entirely different angle---namely, as \textit{an infrastructure that provides services to applications}.

\item End systems attached to the Internet provide an \textbf{Application Programming Interface (API)} that specifies how a program running on one end system asks the Internet infrastructure to deliver data to a specific destination program running on another end system.

\item \textit{A \textbf{protocol} defines the format and the order of messages exchanged between two or more communicating entities, as well as the actions taken on the transmission and/or receipt of a message or other event.}

\item Hosts are sometimes further divided into two categories: \textbf{clients} and \textbf{servers}.

\item Today, the two most prevalent types of broadband residential access are \textbf{digital subscriber line (DSL)} and cable.

\item While DSL makes use of the telco's existing local telephone infrastructure, \textbf{cable Internet access} makes use of the cable television company's existing cable television infrastructure.

\item Although DSL and cable networks currently represent more than 90 percent of residential broadband access in the United States, an up-and-coming technology that promises even higher speeds is the deploy of \textbf{fiber to the home (FTTH)}. As the name suggests, the FTTH concept is simple---provide an optical fiber path from the CO directly to the home.

\item On corporate and university campuses, and increasingly in home settings, a local area network (LAN) is used to connect an end system to the edge router.

\item Examples of physical media include twisted-pair copper, coaxial cable, multimode fiber-optic cable, terrestrial radio spectrum, and satellite radio spectrum. Physical media fall into two categories: \textbf{guided media} and \textbf{unguided media}.

\item Twisted pair consists of two insulated copper wires, each about 1 mm thick, arranged in a regular spiral pattern. The wires are twisted together to reduce the electrical interference from similar pairs close by.

\item Two types of satellites are used in communications: \textbf{geostationary satellites} and \textbf{low-earth orbiting (LEO) satellites}.

\item Most packet switches use \textbf{store-and-forward transmission} at the inputs to the links.

\item If an arriving packet meeds to be transmitted onto a link but finds the link busy with the transmission of another packet, the arriving packet must wait in the output buffer. Thus, in addition to the store-and-forward delays, packets suffer output buffer \textbf{queuing delays}.

\item Since the amount of buffer space is finite, an arriving packet may find that the buffer is completely full with other packets waiting for transmission. In this case, \textbf{packet loss} will occur---either the arriving packet or one of the already-queued packets will be dropped.



\end{itemize}